%%%%%%%%%%%%%%%%%%%%%%%%%%%%%%%%%%%%%%%%%%%%%%%%%%%%%%%%%%%%%%%%%%%%%%%%%%%%%%%%
\section{Results}
%%%%%%%%%%%%%%%%%%%%%%%%%%%%%%%%%%%%%%%%%%%%%%%%%%%%%%%%%%%%%%%%%%%%%%%%%%%%%%%%

%%%%%%%%%%%%%%%%%%%%%%%%%%%%%%%%%%%%%%%%%%%%%%%%%%%%%%%%%%%%%%%%%%%%%%%%%%%%%%%%
\subsection{Creation of data set}
%%%%%%%%%%%%%%%%%%%%%%%%%%%%%%%%%%%%%%%%%%%%%%%%%%%%%%%%%%%%%%%%%%%%%%%%%%%%%%%%

\todo{First focus on the real results, not their quality}

% Simulation of incipient species trees

\todo{redo with new data}

\todo{create figure or table that shows this}

957 out of 960 incipient species trees have been simulated successfully. 
The 3 failed simulations created data files bigger than R can read, 
and have been excluded from further analysis. 

The number of good species ranged from 2 to 14296, with a median of 87 and an average of 503.8. 
There were 6 phylogenies with only 2 good species.  

% Sampling species trees}

\todo{redo with new data}

For each incipient species trees, two species trees were constructed. 
In 653 out of 960 cases, these trees had different branch lengths.
In 302 cases, the trees were identical. In the remaining 5 cases,
there were computational problems (see figure \ref{fig:exit_statuses}).

% Simulation of alignments

\todo{redo with new data}

For each species tree, $n_a$ alignments have been created, 
resulting in $960 \cdot 2 \cdot 2 = 3840$ alignments. 
$?$ of these were informative.

\todo{May result in an overrepresentation of the two same species trees derived from one incipient species tree}

% Simulation of posteriors}

A BEAST2 posterior contains multiple parameter estimates (see figure \ref{fig:ess_distribution} for all),
and each one has an ESS. An ESS of 200 is acceptable for any parameter estimate.
The ESS of the tree likelihood should have such a decent ESS. Figure \ref{fig:ess_distribution_likelihood}
shows the distribution of the likelihood ESS.

\todo{How tp quote https://www.beast2.org/what-is-ess/ ? It states that not all variables
need to have a high ESS, but 'The likelihoods (both of the tree and coalescent model) should have decent ESSs.}

%%%%%%%%%%%%%%%%%%%%%%%%%%%%%%%%%%%%%%%%%%%%%%%%%%%%%%%%%%%%%%%%%%%%%%%%%%%%%%%%
%\begin{figure}[!htbp]
%  \includesvg[width=\textwidth]{figure_ess_distribution_likelihood}
%  \caption{
%    Likelihood ESSes. 
%    The dotted line indicates an 'acceptable' ESS
%  }
%  \label{fig:ess_distribution_likelihood}
%\end{figure}
%%%%%%%%%%%%%%%%%%%%%%%%%%%%%%%%%%%%%%%%%%%%%%%%%%%%%%%%%%%%%%%%%%%%%%%%%%%%%%%%




%%%%%%%%%%%%%%%%%%%%%%%%%%%%%%%%%%%%%%%%%%%%%%%%%%%%%%%%%%%%%%%%%%%%%%%%%%%%%%%%
\subsection{Quantification of the error BEAST2 makes on all trees}
%%%%%%%%%%%%%%%%%%%%%%%%%%%%%%%%%%%%%%%%%%%%%%%%%%%%%%%%%%%%%%%%%%%%%%%%%%%%%%%%

To get a first global impression of the error BEAST2 makes, figure \ref{fig:error}
shows a histogram of the nLTT statistics, colored by the expected mean
duration of speciation. As BEAST2 assumes a mean duration of speciation of zero,
it is expected that the nLTT statistic (a measure of error) is lowest. Zooming
in on the left part of the distribution, there is no clean pattern.
The right part of the tail of distribution shows that the biggest nLTT
statistic values are found for high (but not highest) expected mean durations of
speciation.

\todo{figure \ref{fig:error_for_scr} shows the distribution of errors for all
combinations of PBD parameters}

%%%%%%%%%%%%%%%%%%%%%%%%%%%%%%%%%%%%%%%%%%%%%%%%%%%%%%%%%%%%%%%%%%%%%%%%%%%%%%%%
%\begin{figure}[!htbp]
%  %\def\svgwidth{0.8\textwidth}
%  %\input{figure_error_20171009.pdf_tex}
%  \includesvg[]{figure_error}
%  \caption{
%    Histograms of the nLTT statistic distribution for all simulations
%    The color indicates the expected mean duration of speciation from
%    red (zero) to purple (highest). Similar colors indicate similar
%    values of expected mean duration of speciation.
%  }
%  \label{fig:error}
%\end{figure}
%%%%%%%%%%%%%%%%%%%%%%%%%%%%%%%%%%%%%%%%%%%%%%%%%%%%%%%%%%%%%%%%%%%%%%%%%%%%%%%%

As the expected correlation appears to be absent, figure 
\ref{fig:error_expected_mean_dur_spec}, subfigure $a$, 
shows the correlation between expected
mean duration of speciation and the nLTT statistic. A linear fit shows
that there is indeed an increase in error for an increases expected duration
of speciation. A trendline using locally weighted smoothing, however,
indicates a more complex relationship. Note that this figure only uses
a subset of all measurements, consisting of $n_s$ nLTT statistic values 
per posterior.

Taking the average nLTT value per posterior frees us from computational
constraints and allows us to use the full data set. The resulting
plot, figure \ref{fig:error_expected_mean_dur_spec}, subfigure $b$, shows the same pattern:
A linear fit shows
that there is indeed an increase in error for an increases expected duration
of speciation. But now, a trendline using locally weighted smoothing 
follows this linear relationship closely. 

%%%%%%%%%%%%%%%%%%%%%%%%%%%%%%%%%%%%%%%%%%%%%%%%%%%%%%%%%%%%%%%%%%%%%%%%%%%%%%%%
\begin{figure}[!htbp]
  %\subfloat[]{
  %  \includesvg[width=0.33\textwidth]{figure_error_expected_mean_dur_spec}
  %}
  %\subfloat[]{
  %  \includesvg[width=0.33\textwidth]{figure_error_expected_mean_dur_spec_mean}
  %}
  %\subfloat[]{
  %  \includesvg[width=0.33\textwidth]{figure_error_expected_mean_dur_spec_mean}
  %}
  \caption{
    nLTT statistic distribution
    for different expected mean duration of speciation.
    Subfigure $a$ takes a sample of all $\Delta_{nLTT}$ values, where
    subfigure $b$ uses each posterior's $\mean{\Delta_{nLTT}}$.
    subfigure $c$ uses each posterior's $\median{\Delta_{nLTT}}$.
    The straight blue line shows a linear
    fit, the red line a fit using locally weighted smoothing.
    The dashed horizontal line indicates
    the mean nLTT statistic for the Birth-Death simulations ($SCR = 10^6$).
    Note that this figure uses only a subset of all individually measured
    nLTT statistics
  }
  \label{fig:error_expected_mean_dur_spec}
\end{figure}
%%%%%%%%%%%%%%%%%%%%%%%%%%%%%%%%%%%%%%%%%%%%%%%%%%%%%%%%%%%%%%%%%%%%%%%%%%%%%%%%

The inference error made is expected to be less for a longer DNA
alignment length. 
Figure \ref{fig:error_expected_mean_dur_spec_alignment_length} 
show the correlation between expected mean duration of speciation 
and the nLTT statistic for the two different DNA alignment lengths. 
The first subfigure uses a sample of all $n_s$ individuals nLTT statistics,
where the second subfigure takes only the posterior's 
nLTT statistic mean value, for the complete dataset.

Both figures show again that there is an 
increase in error for an increases expected duration
of speciation, when assuming a linear relation, and
a more complex relationship when using locally weighted smoothing. 
The figures also show that, as expected, 
the error is less with more information. Additionally,
a longer DNA alignment length results in a slower accumulation of error
for increased expected mean duration of speciation.   

%%%%%%%%%%%%%%%%%%%%%%%%%%%%%%%%%%%%%%%%%%%%%%%%%%%%%%%%%%%%%%%%%%%%%%%%%%%%%%%%
\begin{figure}[!htbp]

  %\subfloat[]{
  %  \includesvg[width=0.5\textwidth]{figure_error_expected_mean_dur_spec_alignment_length}
  %}
  %\subfloat[]{
  %  \includesvg[width=0.5\textwidth]{figure_error_expected_mean_dur_spec_mean_alignment_length}
  %}

  \caption{
    Relation between mean nLTT statistic distribution
    and  expected mean duration of speciation, for different DNA
    alignment lengths. Straigh lines show a linear
    fit, the curved lines use a locally weighted smoothing.
    The dashed horizontal lines indicates
    the mean nLTT statistic for the Birth-Death simulations ($SCR = 10^6$)
    for that DNA alignment length.
  }
  \label{fig:error_expected_mean_dur_spec_alignment_length}
\end{figure}
%%%%%%%%%%%%%%%%%%%%%%%%%%%%%%%%%%%%%%%%%%%%%%%%%%%%%%%%%%%%%%%%%%%%%%%%%%%%%%%%

A higher ESS for longer DNA alignments may be an additional
explanation for the observed slower accumulation of error
for increased expected mean duration of speciation.
Figure \ref{fig:ess_expected_mean_dur_spec_alignment_length}
shows that longer DNA alignments have a higher likelihood ESS, 
but this trend is weaker than in figure 
\ref{fig:error_expected_mean_dur_spec_alignment_length} and
the 95\% confidence intervals of the linearily fitted trendlines only
stop overlapping for the highest expected mean duration of speciation.
The locally weighted smoothing trendlines show a different trend,
suggesting that alignment length has no effect on the
likelihood ESS.

\todo{Figure \ref{fig:error_alignment_length} shows the same pattern, seperated
for all combinations of PBD parameters.}

%%%%%%%%%%%%%%%%%%%%%%%%%%%%%%%%%%%%%%%%%%%%%%%%%%%%%%%%%%%%%%%%%%%%%%%%%%%%%%%%
\begin{figure}[!htbp]
  %\includesvg[width=\textwidth]{figure_ess_expected_mean_dur_spec_alignment_length}
  \caption{
    Likelihood ESS and expected mean duration of speciation, for different DNA
    alignment lengths. Straigh lines show a linear
    fit, the curved lines use a locally weighted smoothing.
  }
  \label{fig:ess_expected_mean_dur_spec_alignment_length}
\end{figure}
%%%%%%%%%%%%%%%%%%%%%%%%%%%%%%%%%%%%%%%%%%%%%%%%%%%%%%%%%%%%%%%%%%%%%%%%%%%%%%%%

For each incipient species tree, two species trees were created, by taking
either the youngest or oldest subspecies to represent each species.
The effect of this is shown in figure \ref{fig:error_expected_mean_dur_spec_sampling}.
Both subfigures show the
error made for different $t_{\mean{ds}}$, where the first subfigure uses a sample
of all $\Delta_{nLTT}$ values, where the second subfigure uses the $\mean{\Delta_{nLTT}}$
of each posterior. In both subfigures, the linear and locally weighted smoothing
have overlapping 95\% confidence intervals, hinting that there is no strong effect
of the way subspecies are selected to represent their species.  

%%%%%%%%%%%%%%%%%%%%%%%%%%%%%%%%%%%%%%%%%%%%%%%%%%%%%%%%%%%%%%%%%%%%%%%%%%%%%%%%
\begin{figure}[!htbp]
  %\subfloat[]{
  %  \includesvg[width=0.5\textwidth]{figure_error_expected_mean_dur_spec_sampling}
  %}
  %\subfloat[]{
  %  \includesvg[width=0.5\textwidth]{figure_error_expected_mean_dur_spec_mean_sampling}
  %}
  \caption{
    Likelihood ESS and expected mean duration of speciation, for the two
    different methods to sample species representatives from a subspecies. 
    Straigh lines show a linear
    fit, the curved lines use a locally weighted smoothing.
    See also figure \ref{fig:error_sampling_representative}
  }
  \label{fig:error_expected_mean_dur_spec_sampling}
\end{figure}
%%%%%%%%%%%%%%%%%%%%%%%%%%%%%%%%%%%%%%%%%%%%%%%%%%%%%%%%%%%%%%%%%%%%%%%%%%%%%%%%

\note{Figure \ref{fig:error_sampling_representative} shows the same data, ordered
by the different PBD parameters instead}

The effect of having a low or high tree likelihood ESS is shown in 
figure \ref{fig:figure_error_expected_mean_dur_spec_low_high_ess}. We
define a low ESS as being below median, where a high ESS is defined as 
above-median. Plotting (a sample of or the means of) the nLTT statistic for different mean
duration of speciation for a low an high ESS, shows that there is no clear
difference in error made for posteriors with either low or high tree likelihood ESSes.

%%%%%%%%%%%%%%%%%%%%%%%%%%%%%%%%%%%%%%%%%%%%%%%%%%%%%%%%%%%%%%%%%%%%%%%%%%%%%%
\begin{figure}[!htbp]
  %\subfloat[]{
  %  \includesvg[width=0.5\textwidth]{figure_error_expected_mean_dur_spec_low_high_ess}
  %}
  %\subfloat[]{
  %  \includesvg[width=0.5\textwidth]{figure_error_expected_mean_dur_spec_mean_low_high_ess}
  %}
  \caption{
    Likelihood ESS and expected mean duration of speciation, for the lower and upper half of the tree likelihood ESSes. 
    Straigh lines show a linear
    fit, the curved lines use a locally weighted smoothing.
    See also figure \ref{fig:error_sampling_representative}
  }
  \label{fig:figure_error_expected_mean_dur_spec_low_high_ess}
\end{figure}
%%%%%%%%%%%%%%%%%%%%%%%%%%%%%%%%%%%%%%%%%%%%%%%%%%%%%%%%%%%%%%%%%%%%%%%%%%%%%%%%

It is expected that the bigger phylogeny is simulated, the less the error
will be. Figure \ref{fig:figure_error_tree_size} shows the relation between the
number of taxa and the mean nLTT statistic. Although most simulated
phylogenies have less than 250 taxa, there is a clear positive trend, 
which is the opposite of the
expactation: the bigger the phylogeny, the bigger the error made.
As the nLTT statistic normalized for the number of taxa, some other
mechanism must be involved. 

%%%%%%%%%%%%%%%%%%%%%%%%%%%%%%%%%%%%%%%%%%%%%%%%%%%%%%%%%%%%%%%%%%%%%%%%%%%%%%%%
\begin{figure}[!htbp]
  %\subfloat[]{
  %  \includesvg[width=0.5\textwidth]{figure_error_tree_size}
  %}
  %\subfloat[]{
  %  \includesvg[width=0.5\textwidth]{figure_error_tree_size_mean}
  %}
  \caption{
    The effect of tree size on nLTT.
    Straigh lines show a linear
    fit, the curved lines use a locally weighted smoothing.
    The linear fit and $R^2$ valaues are shown
  }
  \label{fig:figure_error_tree_size}
\end{figure}
%%%%%%%%%%%%%%%%%%%%%%%%%%%%%%%%%%%%%%%%%%%%%%%%%%%%%%%%%%%%%%%%%%%%%%%%%%%%%%%%

For each species tree, $n_a$ different alignments are simulated. For each of
these alignments, $n_b$ posteriors are simulated, using the same alignment. 
It is assumed that two posteriors from the same alignment
give rise to the same parameter estimates distributions.
distribution. Figure \ref{fig:figure_posterior_distributions_likelihood} 
shows the distribution of Mann-Whitney U test p-values for each pair of 
posteriors working on a same alignment, 
using their tree likelihood estimates.

The figure shows that the tree likelihood posterior distributions 
from a same alignment nearly always
are significantly different. Subfigure (b) and (c) show the distribution of
tree likelihood estimates for a posterior pair with the lowest (b) and 
highest (c) p-value.

%%%%%%%%%%%%%%%%%%%%%%%%%%%%%%%%%%%%%%%%%%%%%%%%%%%%%%%%%%%%%%%%%%%%%%%%%%%%%%%%
\begin{figure}[!htbp]
  %\subfloat[]{
  %  \includesvg[width=0.8\textwidth]{figure_posterior_distribution_likelihoods}
  %}
  %\subfloat[]{
  %  \includesvg[width=0.5\textwidth]{figure_posterior_distribution_likelihoods_low}
  %}
  %\subfloat[]{
  %  \includesvg[width=0.5\textwidth]{figure_posterior_distribution_likelihoods_high}
  %}
  \caption{
    (a) Mann-Whitney U test p-values between $n_b$ posteriors working on the same DNA alignment
    for tree likelihood estimates.
    Low p-values indicate the distributions are different.
    High p-values indicate the distributions can be rejected to be different.
    The vertical dotted line shows the p-value of 0.05.
    (b) and (c): distribution of tree likelihoods for the two posteriors that had the lowest
    (b) and highest (c) p-value.
  }
  \label{fig:figure_posterior_distributions_likelihood}
\end{figure}
%%%%%%%%%%%%%%%%%%%%%%%%%%%%%%%%%%%%%%%%%%%%%%%%%%%%%%%%%%%%%%%%%%%%%%%%%%%%%%%%

Posteriors based on the same alignment are also (next to estimated tree likelihood) 
assumed to have a similar nLTT statistic distribution. 
Figure \ref{fig:figure_posterior_distribution_nltt} 
shows the distribution of Mann-Whitney U test p-values for each pair of 
posteriors working on a same alignment, 
using their nLTT statistics.

The figure shows that the nLTT statistic distributions 
from a same alignment are commonly different and commonly the same. 
Subfigures $(b)$ and $(c)$ show the distribution of
nLTT statistics for a posterior pair with the lowest $(b)$ and 
highest $(c)$ p-value.

%%%%%%%%%%%%%%%%%%%%%%%%%%%%%%%%%%%%%%%%%%%%%%%%%%%%%%%%%%%%%%%%%%%%%%%%%%%%%%%%
\begin{figure}[!htbp]
  %\subfloat[]{
  %  \includesvg[width=0.8\textwidth]{figure_posterior_distribution_nltts}
  %}
  %
  %\subfloat[]{
  %  \includesvg[width=0.5\textwidth]{figure_posterior_distribution_nltts_low}
  %}
  %\subfloat[]{
  %  \includesvg[width=0.5\textwidth]{figure_posterior_distribution_nltts_high}
  %}
  \caption{
    $(a)$ Mann-Whitney U test p-values between $n_b$ posteriors working on the same DNA alignment
    for nLTT statistics.
    Low p-values indicate the distributions are different.
    High p-values indicate the distributions can be rejected to be different.
    The vertical dotted line shows the p-value of 0.05.
    $(b)$ and $(c)$: distribution of nLTT statistics for the two posteriors that had the lowest
    $(b)$ and highest $(c)$ p-value.
  }
  \label{fig:figure_posterior_distribution_nltt}
\end{figure}
%%%%%%%%%%%%%%%%%%%%%%%%%%%%%%%%%%%%%%%%%%%%%%%%%%%%%%%%%%%%%%%%%%%%%%%%%%%%%%%%

The nLTT statistic normalizes a phylogeny to have a crown age of 1.
To prevent overlooking patterns in posterior crown ages, 
figure \ref{fig:figure_posterior_distribution_crown_ages}
shows the distribution of crown ages of all simulations.
Almost all simulations have a crown age close to $0.15$, but there
are rare cases with crown ages above $0.2$, ranging to $83.4$.

\todo{I think it is very suspicious that the range centers on 0.15, when
the crown age is 15. Check this!}

%%%%%%%%%%%%%%%%%%%%%%%%%%%%%%%%%%%%%%%%%%%%%%%%%%%%%%%%%%%%%%%%%%%%%%%%%%%%%%%%
\begin{figure}[!htbp]
  %\subfloat[]{
  %  \includesvg[width=0.8\textwidth]{figure_posterior_distribution_crown_ages}
  %}
  %
  %\subfloat[]{
  %  \includesvg[width=0.5\textwidth]{figure_posterior_distribution_crown_ages_low_count}
  %}
  %\subfloat[]{
  %  \includesvg[width=0.5\textwidth]{figure_posterior_distribution_crown_ages_high_count}
  %}
  \caption{
    $(a)$ Distribution of crown ages in all posteriors. 
    The horizontal line indicates the maximum count for subfigure $(b)$. 
    The vertical line shows the maximum crown age for subfigure $(c)$.
    $(b)$ Rare crown ages
    $(c)$ Common crown ages
  }
  \label{fig:figure_posterior_distribution_crown_ages}
\end{figure}
%%%%%%%%%%%%%%%%%%%%%%%%%%%%%%%%%%%%%%%%%%%%%%%%%%%%%%%%%%%%%%%%%%%%%%%%%%%%%%%%

Again, it is assumed that posteriors based on the exact same alignment
have a similar crown age distribution. Of all posterior pairs, the Mann-Whitney
U test p-values are shown in figure \ref{fig:figure_posterior_distribution_crown_ages_p_values}.
It shows that crown age distributions being different is common and these
being the same is uncommon. 

%%%%%%%%%%%%%%%%%%%%%%%%%%%%%%%%%%%%%%%%%%%%%%%%%%%%%%%%%%%%%%%%%%%%%%%%%%%%%%%%
\begin{figure}[!htbp]
  %\subfloat[]{
  %  \includesvg[width=0.8\textwidth]{figure_posterior_distribution_crown_ages_p_values}
  %}
  %
  %\subfloat[]{
  %  \includesvg[width=0.5\textwidth]{figure_posterior_distribution_crown_ages_lowest_p_value}
  %}
  %\subfloat[]{
  %  \includesvg[width=0.5\textwidth]{figure_posterior_distribution_crown_ages_highest_p_value}
  %}
  \caption{
    $(a)$ Mann-Whitney U test p-values comparing the distribution of crown ages
    between each posterior pair.
    $(b)$ Crown age distributions of a posterior pair with the lowest p-value
    $(c)$ Crown age distributions of a posterior pair with the highest p-value
  }
  \label{fig:figure_posterior_distribution_crown_ages_p_values}
\end{figure}
%%%%%%%%%%%%%%%%%%%%%%%%%%%%%%%%%%%%%%%%%%%%%%%%%%%%%%%%%%%%%%%%%%%%%%%%%%%%%%%%

It is expected that if a tree is created by a Birth-Death process, BEAST2 is able
to reasonably estimate those parameters.

%a) `figure_posterior_distribution_bd_bd` histogram of BirthDeath, vertical line at simulation value	
%b) `figure_posterior_distribution_bd_br2` histogram of birthRate2, vertical line at simulation value	
%c) `figure_posterior_distribution_bd_rdr2` histogram of relativeDeathRate2, vertical line at simulation value
%figure_posterior_distribution_bd



% Bad species tree:
% figure_error_posterior_nltt_bad_stree
% figure_error_posterior_nltt_bad_post_tree


