\section{Discussion}

This article has quantified the error made in inference when (1) nature would follow the
protracted speciation model, and (2) her DNA sequences would be analysed with a birth-death
model.

% The PBD model is just one implementation of speciation taking time

The key feature of the PBD model is that it encompasses speciation taking time in the
simplest way possible (that allows for mathematical analysis). The PBD model allows for
multiple incipient species stages (where its earliest version had infinite
stages \cite{rosindell2010protracted}). This research assumes that there is only one
incipient species stage. One could argue that speciation takes place in more
biological stages \todo{Add reference}, but we tried to use the simplest incarnation of the PBD possible.
Also, the PBD model can also be viewed as a model of species recognition \todo{?Purvis et al., 2009},
in which (human) recognition needs to build up, which we consider to be sufficiently
encapsulated in one single stage. Future research may investigate if adding multiple incipient
species stages for simulating nature changes the error made by (non-protracted) birth-death models.

% Equal and constant rates

The PBD model allows for multiple incipient species stages. 
Each stage has its own extinction and speciation-initiation and 
speciation-completion rate (except for good species, which lacks 
a speciation-completion rate). These rates are assumed to be constant
in time and equal for all stages. We chose to use the constant-rate
birth-death model to keep this investigation as simple as possible.
Future research may investigate whether a biologically more realistic
rate model, for example, by adding diversity-dependence, results in
the same inference error.

We chose to set the extinction and speciation-initiation rates equal for all stages,
again for choosing the simplest model. We expect these rates to differ in reality:
incipient species have low initial abundances, which has an effect on extinction
and speciation-initiation rates. 

In this research, we nevertheless set these rates equal. 
Not only because it helps constrain the parameter space investigated,
also paraphylies would pop up in the simulated trees. This would violate more assumptions
of the BEAST2 model used as a prior and increase the complexity of the analysis.

% Sampling species trees from incipient species trees

An incipient species tree may have multiple incipient lineages from the
same ancestor. We constructed two species trees from each incipient species
tree, by selecting the youngest and oldest subspecies to represent a species.
This selection maximizes the chance the species trees are different, where
random sampling of representatives would also be a valid approach, closer
to what would happen when scientists take measurements in the field. Our
approach increases the error made.

% Nucleotide substitution model

For the simulation of DNA alignments, the Jukes-Cantor nucleotide substitution model
is used. This substitution model is a simplification of the DNA mutation processes
happening in reality \todo{Add reference}. In this simulation research, however, we could also
state in BEAST2 that this nucleotide substitution model is used. Thus, we kept
the simulation of the nucleotides in time as simple as possible, and used that
knowledge in later steps. We expect that any nucleotide substitution model, when
used in both alignment generation and in phylogeny inference, will do equally well,
except that the full computational pipeline would take needlessly longer.

% Allopatric speciation only

When a speciation event is initiated, this investigation assumes no gene flow
between the two incipient lineages. This follows the allopatric mode of speciation, 
which is assumed to be most prominent of nature \todo{Add reference}. It would be interesting to
allow gene flow between related incipient species, 
so also peripatric and sympatric modes of speciation would be allowed for.
We predict that this would increase the error in inference.

% Why not put the tree prior into an MCMC tool directly?
One could argue we should have written a PBD tree prior for an MCMC phylogenetic tool.
We did not do that for two reasons. First, we think it is rational to first
measure if there is need for a more complex tree prior. Second, 
we could not have done so, as a phylogenetic tool like BEAST2 assumes species
trees, where our input consists of incipient species trees. Before this
research, we were unsure how to sample species representatives. This research
has shown that any sampling is just as good, opening up avenues for
implementing a PBD species tree prior.

% BEAST2 is just one of many tools

BEAST2 is one of many MCMC phylogenetic tools with its
own internal technicalities. Similar tools are the R phytools
package \cite{phytools}, BAli-Phy \cite{suchard2006bali}
and BayesPhylogenies \cite{pagel2007bayesphylogenies}, to name a few.
BEAST2 was chosen, because (1) it's FOSS, (2) its software
architecture is modular, encouraging extension, (3) its development
team follows good practices in software development.


% BEAST2 parameter settings: strict clock

\todo{elaborate}

Use of strict clock is clean, but unrealistic.

\todo{elaborate}

% Alignment generation

Use of JC69 is rational, but unrealistic.

% Use for empiricists

\todo{elaborate}

This reseach shows that a DNA sequence length of 10k [is good].
[must be effective DNA sequence length]
[no indels generated]

% Low number of incipient species trees

\todo{elaborate}

For each point in parameter space, twenty incipient species tree are created.
Was fitted to 1000 jobs.

% Failed simulations

Not all parameter combinations resulted in posteriors with a high ESS,
due to computational constraints. We think this is a valid approach:
would a phylogeny get too big, we as scientists would cut it to
pieces that are analytically tractable. Because these simulations
are only a minority (as shown in the Results section), we do not
expect this to influence our conclusions.

% nLTT statistic is just one of many summary statistics

We used the nLTT statistic for quantifying the error between the
true and inferred phylogeny. There are simpler statistics that
can do the same, like the gamma \cite{pagel1999inferring}, 
delta-r \cite{pigot2010shape} \todo{check citation}, 
tree balance \todo{Add reference} or
Normalized Rooted Branch Score [Drummond] statistic, 
or more complex, like the spectral analysis [Hendy and Penny, 1993].
We consider those other summary statistics equally good, we just picked the
one we thought of the most informative per unit of added computational time.

% Add:
%
% Computational Performance and Statistical Accuracy of *BEAST and Comparisons with Other Methods (by Huw A. Ogilvie Joseph Heled Dong Xie Alexei J. Drummond)
% cites rooted branch score, RBS, as being from Heled and Bouckaert 2013, `Looking for trees in the forest: Summary tree from posterior samples`
