\section{Introduction}

% Biology: speciation takes time

It takes time for new species-to-be to achieve the status of a new species.
First, it takes time for the new species-to-be has to acquire sufficient different mutations to 
establish reproductive isolation \cite{schluter2001ecology}. Second, 
it takes time for us to recognize reproductive isolation
and reclassify that one single species has become two (or more) species. As an
example, \cite{fennessy2016multi} showed that Africa has four instead of only one
giraffe species, with a common ancestor estimated at around 2 million years ago.

% The use of speciation models

From DNA sequences obtained in the present, we can construct phylogenies
to infer the evolutionary relationship between species and the time 
from which ancestral species developed into separate.
  
As we only sample DNA of extant species, 
we can only infer a reconstructed tree, in which
the lineages that have gone extinct are absent.
Although most models assume extinctions (\cite{yule1925mathematical} being the
classic exception, as at its time computation was expensive), 
information on extinction times is lost due to our sampling.


The time when reproductive isolation started to take off, is
also lost in our sampling method, if the new species-to-be is not 
recognized yet as such. Without fossil date, one cannot observe the
possibly many times that reproductive isolation built up, but was lost again.

% First model: constant rate birth death model

Although we know that speciation takes time, we commonly ignore this when
constructing a phylogeny, by choosing a constant-rate birth-death model
as a speciation model. This can be justified, as for lineages 
that have speciated far back in the past, there is no error. 
Yet in the present, we do not recognize the future species-to-be
that are already genetically converging, resulting in an underestimation
of the number of species present today, which can only be 
concluded in retrospect.

The constant-rate birth-death model (as described in for example \cite{nee1994reconstructed}) 
is among the simplest speciation models, and assumes a constant speciation
 rate $\lambda$ and constant extinction rate $\mu$.

Additionally, it assumes speciation is instantaneous.
The constant-rate birth-death model is popular for its simplicity, yet
has also served as a starting point for more elaborate speciation models.

% Other non-protracted speciation models

Other speciation models may assume that speciation rate changes in 
time \cite{rabosky2008explosive}, is dependent on the amount of species 
present \cite{etienne2011diversity}, or is trait dependent \cite{fitzjohn2009estimating}.

In these examples, the original birth-death model has been extended by making 
speciation rate dependent on time, diversity or trait value respectively.
The assumption that all these extensions share is that speciation is instantaneous;
that after a new branching event, both two lineages are (or are recognized as being) different
species.

% Protracted speciation model

% Good and incipient states

The protracted speciation model \cite{etienne2012prolonging} allows for speciation taking time.

It adds an additional species state (see also figure \ref{fig:pbd_states}),
coined the 'incipient' stage, which a lineages has to complete before becoming
a 'good' species.

One view is to say that good species have achieved reproductive isolation,
where incipient species are in the process of achieving this \todo{Add reference}.

As genetic drift is unlikely to be a relevant speciation mechanism,
this is more likely to be ecological speciation [Sobel et al., 2009]. 
\todo{RSE: Ben je het daarmee eens? Zou drift niet een belangrijke ooraak van speciation op islands kunnen zijn bijvoorbeeld?}

Alternatively, an incipient species can be described as a good-species-to-be,
yet not recognized as such \todo{?[Purvis et al., 2009]?}.

%%%%%%%%%%%%%%%%%%%%%%%%%%%%%%%%%%%%%%%%%%%%%%%%%%%%%%%%%%%%%%%%%%%%%%%%%%%%%%%%
\begin{figure}
  \centering
  \begin{tikzpicture}[->,>=stealth',shorten >=1pt,auto,node distance=4cm, semithick]   
  \tikzstyle{every state}=[]
  \node[state] (A)              {Good};   
  \node[state] (B) [right of=A] {Incipient};   
  \node[state] (C) [below of=B] {Extinct};   
  \path (A) edge [bend right] node {$b_g$} (B)
        (A) edge [bend right] node {$\mu_g$} (C)
        (B) edge [loop above] node {$b_i$} (B)
        (B) edge [bend right] node {$\lambda$} (A)
        (B) edge [bend right] node {$\mu_i$} (C); 
  \end{tikzpicture}

  \caption{
    The states and transitions of a species in the PBD model.
    $b_i$: speciation-initiation rate of incipient species. 
    $b_g$: speciation-initiation rate of good species. 
    $\lambda$: speciation completion rate. 
    $\mu_i$: extinction rate of incipient species. 
    $\mu_g$: extinction rate of good species. 
    Figure after Etienne et al, 2014, Evolution
  }
  \label{fig:pbd_states}
\end{figure}
%%%%%%%%%%%%%%%%%%%%%%%%%%%%%%%%%%%%%%%%%%%%%%%%%%%%%%%%%%%%%%%%%%%%%%%%%%%%%%%%

% Biological mechanism of reproductive isolation

The protracted speciation model makes no assumption about the exact biological
mechanism by which reproductive isolation is 
acquired \cite{etienne2012prolonging, etienne2014estimating,rosindell2010protracted}.

A 'good' state can be viewed as a state in which there is convergent selection
for that lineage.

The 'incipient' state can be seen as a state in which there is the potential
for speciation.

A first example can be from the BDM \todo{Add reference} model, the state in which there
are all three genotypes, before the intermediate genotype is lost.
A second example can be in a sexual selection model [van Doorn and Weissing, 2002], 
where there is the onset for convergent selection.
In both cases, these intermediate states may fall back to the original
state of one species, or complete to produce a new species.

% Effects on topology

A feature of the protracted birth-death model, is that it may result in
paraphylies when the rates between good and incipient species differ \todo{Add reference}.
Paraphylies may be more than an experimental artifact and are claimed to
be an inherent feature of nature \cite{funk2003species}.

\todo{Add something like: this research solves this by sampling from paraphyletic trees (at the subspecies level) to create non-paraphyletic trees at the species level}

% Good and incipient rates

Where the classic birth death model assumes constant speciation and extinction
 rates, the protracted birth-death model allows these rates to be state
 dependent.

The only additional parameter is the rate at which incipient species become
(recognized as) good species.
When setting that parameter to infinity, incipient species become good
species instantaneous and the model falls back to a constant-rate (non-protracted) birth-death model \todo{Add reference}.

% Relationship to constant-rate birth-death model

\todo{?Wat bedoelde ik met deze onzin?}
Although both standard and protracted birth-death models use a parameter 
$\lambda$, the $\lambda$ in the standard birth death model denotes
a speciation rates (initiation and completion), where in the PBD model
$\lambda$ denotes only the speciation initiation rate. 
There is no relationship known between the parameters of the constant-rate
birth-death model and the protracted birth-death model \cite{etienne2014estimating} 
\todo{refesh memory, then update paragraph}.

% Good and incipient rates do not need to be constant

The assumption that speciation takes time is independent of the dynamics
of the speciation and extinction rates.
The protracted speciation model in this research assumes these rates are
constant, but these rates can easily be made to depend on time, diversity
or trait.

% Use of a speciation model in inferring a phylogeny

Speciation models are widely used to make inferences from genetic data.
There are multiple computer programs to create phylogenies and/or parameter
estimates from aligned DNA sequences.
BEAST2 is a widely used tool that allows for a Bayesian approach to 
phylogenetics \cite{bouckaert2014beast}.

BEAST2 supplies the user with multiple speciation models, that all assume
instantaneous speciation.
 
% This study

This simulation investigates the consequence of BEAST2 using instantaneous
speciation, by simulating a 'true' tree that is protracted, simulate DNA
alignments following that tree, and seeing how well BEAST2 can recover
the original phylogeny.
 
% Novelty

This study is the first to measure the error made when acknowledging that
speciation takes time, yet using one of the many tools that ignores this
fact.



Part of the parameter space will create 'true' trees that are not protracted.
This will result in a test of BEAST2 against itself, which is, to the best
of our knowledge, not been done, probably due to the extensive computations
that are needed \todo{RSE: Heb je dit uitgebreid getest? Misschien zou je Remco kunnen vragen}.

% Analysis

The 'true' and inferred trees are compared with the nLTT statistic, which
is a novel and efficient summary statistic, that has been proven to have
better performance than the gamma statistics in Approximate
Bayesian Computation \cite{janzen2015approximate} (yet on par with the likelihood).

The nLTT statistics is the summed absolute difference between two 
Lineages-Through-Time plot lines that are normalized to 
let both time and number of lineages
range from zero to one.
Additionally, this statistic allows to pinpoint the time span that contributed
most to its value.

% Expectations

It is expected that for higher protractedness (thus deviation from BEAST2
its assumptions), the error will increase.
Trees are expected to be inferred better, when there is more information
available to construct them, thus we expect less error when there are more
taxa and when the simulation DNA alignments are longer.
The error is expected to be biggest close to the present, as there will
be the most incipient species present.
 
It is unknown, however, under which biological settings this error is relevant.

% Preview results

This study shows that ... \todo{put result here}.
