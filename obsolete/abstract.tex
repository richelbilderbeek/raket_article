The construction of phylogenies helps us answer evolutionary biological
questions.
Our current phylogenetic tools ignore the fact that speciation takes time,
which has an effect unknown in phylogeny reconstruction.
Here, we simulate true incipient species trees and their corresponding
DNA alignments, from which we measure the reconstruction of the phylogeny
using a standard birth-death model.
We measure the errors that the Bayesian phylogenetic software tool BEAST2
gives when recovering simulated phylogenies, for different times-to-speciate,
under a range of additional parameter settings.
It has been found that branch lengths are consistently and strongly 
underestimated for biologically relevant parameters.
This research shows that protractedness is a complexity of nature that
cannot always be ignored and should be incorporated in our phylogenetic
tools.
