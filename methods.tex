%%%%%%%%%%%%%%%%%%%%%%%%%%%%%%%%%%%%%%%%%%%%%%%%%%%%%%%%%%%%%%%%%%%%%%%%%%%%%%%%
\section{Methods}
%%%%%%%%%%%%%%%%%%%%%%%%%%%%%%%%%%%%%%%%%%%%%%%%%%%%%%%%%%%%%%%%%%%%%%%%%%%%%%%%

This research measures the error our current BD tools make, when
the real process has been protracted. Figure \ref{fig:experiment} shows
the following overview of the experiment graphically. 
An in silico data set is created for a wide
range of parameters from which incipient species trees are simulated (A in figure \ref{fig:experiment}). 
These phylogenies serve as 'the truth' or 'nature'. From that incipient species tree,
species trees are created, by picking one representive per species.
This is done twice, as either the youngest (B1) or oldest (B2) subspecies is chosen
to represent a species. Per species tree, $n_a$ alignments are simulated (C1 and C2).
Per alignment, $n_b$ posteriors are inferred, 
each consisting of $n_s$ phylogenies and parameter estimates (D1 and D2). Each
species tree is then compared with the species tree its alignment is based on (for
example, B1 and D1).

%%%%%%%%%%%%%%%%%%%%%%%%%%%%%%%%%%%%%%%%%%%%%%%%%%%%%%%%%%%%%%%%%%%%%%%%%%%%%%%%
\begin{figure}
  \centering 
  \resizebox {0.8\textwidth} {!} {
    \CreateTikzFigureExperiment{} 
  }
  \caption{
    Overview of the experiment: 
    A: per parameter set, there are $n_i$ incipient species trees created. 
    B1: constructing a species trees by using the youngest subspecies $A1$ to represent good species $A$.
    B2: constructing a species trees by using the oldest subspecies $A2$ to represent good species $A$.
    C1: constructing one of the $n_a$ alignments of B1. 
    C2: constructing one of the $n_a$ alignments of B2. 
    D1: reconstructing a species trees from alignment B1.
    D2: reconstructing a species trees from alignment B2.
    Comparisons are done between a species tree and its reconstructed species tree, 
    for example, between B1 and D1.
    All phylogenies are drawn at the same scale.
  }
  \label{fig:experiment}
\end{figure}
%%%%%%%%%%%%%%%%%%%%%%%%%%%%%%%%%%%%%%%%%%%%%%%%%%%%%%%%%%%%%%%%%%%%%%%%%%%%%%%%

%%%%%%%%%%%%%%%%%%%%%%%%%%%%%%%%%%%%%%%%%%%%%%%%%%%%%%%%%%%%%%%%%%%%%%%%%%%%%%%%
\subsection{Parameter space}
\label{subsec:parameter_space}
%%%%%%%%%%%%%%%%%%%%%%%%%%%%%%%%%%%%%%%%%%%%%%%%%%%%%%%%%%%%%%%%%%%%%%%%%%%%%%%%

%%%%%%%%%%%%%%%%%%%%%%%%%%%%%%%%%%%%%%%%%%%%%%%%%%%%%%%%%%%%%%%%%%%%%%%%%%%%%%%%
\begin{table}
  \centering 
  \begin{tabular}{l l}
    \hline
    Symbol & Description \\
    \hline
    \hline
    $b_g$ & Speciation initiation rate of a good species (probability per lineage, per million year) \\
    $b_i$ & Speciation initiation rate of an incipient species (probability per lineage, per million year) \\
    $\lambda$ & Speciation completion rate (probability per lineage, per million year) \\
    $\mu_g$ & Extinction rate of a good species (probability per lineage, per million year) \\
    $\mu_i$ & Extinction rate of an incipient species (probability per lineage, per million year) \\
    \hline
    $t_c$ & Crown age (million years) \\
    $r$ & Mutation rate (probability of mutation per nucleotide per million years) \\
    $l_a$ & DNA alignment length (base pairs) \\
    \hline
    $n_i$ & Number of incipient species trees per parameter set \\
    $n_a$ & Number of alignments per species tree \\
    $n_b$ & Number of BEAST2 runs per alignment \\
    \hline
    $n_s$ & Number of MCMC samples \\
    $i_s$ & MCMC sampling interval \\
    \hline
    $n_{ti}$ & Number of taxa in an incipient species tree \\
    $n_{t}$ & Number of taxa in a (good) species tree \\
    $t_{\mean{ds}}$ & Average duration of speciation \\
    $t_{sc}$ & Speciation completion time \\
    $\Delta_{nLTT}$ & nLTT statistic value \\
    $\mean{\Delta_{nLTT}}$ & Average nLTT statistic value of a posterior \\
    $\median{\Delta_{nLTT}}$ & Median nLTT statistic value of a posterior \\
    \hline
  \end{tabular}
  \caption{
    Parameter descriptions. The sections are (1) the biological parameters, (2) the
    biological tuning parameters, (3) parameters involved in replication, (4) MCMC
    tuning parameters, and (5) other symbols.
  }
  \label{table:parameter_descriptions}
\end{table}
%%%%%%%%%%%%%%%%%%%%%%%%%%%%%%%%%%%%%%%%%%%%%%%%%%%%%%%%%%%%%%%%%%%%%%%%%%%%%%%%

\todo{@RSE: would you prefer $t_{\mean{ds}}$ or $\mean{t_{ds}}$ for average duration of speciation?}

%%%%%%%%%%%%%%%%%%%%%%%%%%%%%%%%%%%%%%%%%%%%%%%%%%%%%%%%%%%%%%%%%%%%%%%%%%%%%%%%
\begin{table}
  \centering 
  \begin{tabular}{l l}
    \hline
    Parameter             & Values \\
    \hline
    \hline
    $b = b_g = b_i$       & 0.1, 0.5, 1.0 \\
    $\lambda$             & 0.1, 0.3, 1.0, $10^6$ \\
    $\mu = \mu_g = \mu_i$ & 0.0, 0.1, 0.2, 0.4 \\
    \hline
    $t_c$                 & 15 \\
    $r$                   & 0.01 \\
    $l_a$                 & $10^3$, $10^4$ \\
    \hline
    $n_i$                 & 20 \\
    $n_a$                 & 2 \\
    $n_b$                 & 2 \\
    \hline
    $n_s$                 & $10^3$ \\
    $i_s$                 & $10^3$ \\
    \hline
  \end{tabular}
  \caption{
    Parameter values. The sections are (1) the biological parameters, (2) the
    biological tuning parameters, (3) parameters involved in replication, 
    and (4) MCMC tuning parameters.
  }
  \label{table:parameter_values}
\end{table}
%%%%%%%%%%%%%%%%%%%%%%%%%%%%%%%%%%%%%%%%%%%%%%%%%%%%%%%%%%%%%%%%%%%%%%%%%%%%%%%%

% Biological parameters

This study primarily investigates combinations of the five biological 
parameters ($b_g$, $b_g$, $\lambda$, $\mu_g$ and $\mu_g$, 
see also figure \ref{fig:pbd_states} and table \ref{table:parameter_descriptions}), 
in a close to full factorial fashion, with two exceptions.
First, following \cite{etienne2014estimating}, birth and extinction rates were each 
set equal between good and incipient species ($b_g = b_i$, and $\mu_g = \mu_i$).
Second, only parameter combinations with speciation initiation rates greater
than extinction rates were used ($b_g > \mu_g$). The values 
used in the experiment (see table \ref{table:parameter_values}) 
were selected to have a reasonable expected number of incipient 
lineages, $E(n_{ti})$, where \todo{?too basic to have a reference to? I assume Kendall, 1948, \cite{kendall1948generalized}, but could not look it up here and now}: 

\begin{equation}
  E(n_{ti}) = e^{ (b - \mu) \cdot t_c }
\end{equation}

One value of speciation completion rate $\lambda$ is set (close to) infinity, as
then the protracted speciation model falls back to a constant-rate 
(non-protracted) birth-death model [?Etienne et al., 2012 or already Rosindell and Etienne 2011?].

% Biological tuning parameters

Secondarily, there are three biological tuning parameters. The crown age of
the incipient species tree, $t_c$, is set to 15 million years for all simulations, 
following \cite{etienne2014estimating}. The per-nucleotide mutation rate, $r$, and the
DNA aligment length, $l_a$, together must be tuned to result in informative
DNA alignments (see section \ref{subsec:simulating_alignments}). $r$ is identical in all
simulations. Increasing $l_a$ results in more informative alignments, and this
effect is measured by using two values of $l_a$ 
per biological parameter combination.

% Replication parameters

All processes involving stochasticity have replicates. 
For each set of parameters, $n_i$ incipient species trees are simulated. 
From each species tree, $n_a$ alignments are simulated. From each alignment, 
$n_b$ posteriors are generated.

% MCMC tuning parameters

The posteriors are generated using BEAST2, which uses a Monte-Carlo Markov-Chain (MCMC)
algorithm. This algorithm is tuned to result in a high effective sample size (ESS) by
two constants. The first constant is the MCMC sampling interval, which is $i_s$ states.
$i_s$ which must be high enough to avoid correlation between subsequently sampled posterior states.  
The second constant is the number of MCMC state samples, $n_s$, which equals the ESS iff \todo{@RSE: Can I use iff?}
$i_s$ is high enough to have no correlation between subsequently sampled posterior states.

%%%%%%%%%%%%%%%%%%%%%%%%%%%%%%%%%%%%%%%%%%%%%%%%%%%%%%%%%%%%%%%%%%%%%%%%%%%%%%%%
\subsection{Simulating incipient species trees}
\label{subsec:simulating_incipient_species_trees}
%%%%%%%%%%%%%%%%%%%%%%%%%%%%%%%%%%%%%%%%%%%%%%%%%%%%%%%%%%%%%%%%%%%%%%%%%%%%%%%%

From each parameter set, $n_i$ incipient species trees with a crown age
of $t_c$ and at least two extant taxa, 
are created in R \cite{R} using the PBD package \cite{PBD}.
The values of the biological parameters encourage that the number of tips in
each phylogeny is kept within computational limits. However, due to
the stochastic nature of the Doob-Gillespie algorithm used in the creation
of these phylogenies, phylogenies can, by chance, become very big. 
There is no maximum number of taxa set, yet imposed instead by the R 
programming language itself, due to a limit on the amount of memory that can be 
allocated. The incipient species trees that could not be simulated are
counted and discarded from further analysis.  

%%%%%%%%%%%%%%%%%%%%%%%%%%%%%%%%%%%%%%%%%%%%%%%%%%%%%%%%%%%%%%%%%%%%%%%%%%%%%%%%
\subsection{Sampling species trees}
\label{subsec:sampling_species_trees}
%%%%%%%%%%%%%%%%%%%%%%%%%%%%%%%%%%%%%%%%%%%%%%%%%%%%%%%%%%%%%%%%%%%%%%%%%%%%%%%%

Per incipient species tree, two species trees are created. 
To create a species tree (a tree in which each species is represented once)
for an incipient species tree, one representative for each species is picked. 
If a species has initiated one or more incipient species, either the youngest
or the oldest representative is picked. See figure \ref{fig:sampling} for an example.

An alternative sampling method would be to pick 
random representatives of each species. This would decrease the variability
between the two species tree. Maximizing variability between the two
species trees allows for clearer measuments on the effect of sampling.   

%%%%%%%%%%%%%%%%%%%%%%%%%%%%%%%%%%%%%%%%%%%%%%%%%%%%%%%%%%%%%%%%%%%%%%%%%%%%%%%%
\begin{figure}
  \centering 
  \CreateTikzFigureSampling{}
  \caption{
    Sampling an incipient species tree. 
    A) At stem age $t_0$, $A1$ is present.
    At crown age $t_1$, its derived incipient species $A2$ is born. 
    At $t_2$, $A2$ itself speciates into a new (unnamed) species, 
    that becomes good species $B$ at $t_3$. At present time, $t_4$, there
    exist two possible representatives of species $A$. 
    B) Using the oldest representative, $A1$, to represent species $A$, the crown
    age of the resulting species tree lies at $t_1$. 
    C) Using the youngest representative, $A2$, to represent species $A$, the crown
    age of the resulting species tree lies at $t_2$. 
  }
  \label{fig:sampling}
\end{figure}
%%%%%%%%%%%%%%%%%%%%%%%%%%%%%%%%%%%%%%%%%%%%%%%%%%%%%%%%%%%%%%%%%%%%%%%%%%%%%%%%

%%%%%%%%%%%%%%%%%%%%%%%%%%%%%%%%%%%%%%%%%%%%%%%%%%%%%%%%%%%%%%%%%%%%%%%%%%%%%%%%
\subsection{Simulating alignments}
\label{subsec:simulating_alignments}
%%%%%%%%%%%%%%%%%%%%%%%%%%%%%%%%%%%%%%%%%%%%%%%%%%%%%%%%%%%%%%%%%%%%%%%%%%%%%%%%

From each species tree, $n_a$ DNA alignments were simulated that match the
phylogeny's structure, using the phangorn R package \cite{phangorn}. 
Each alignment contains one DNA sequence per species 
of $l_a$ nucleotides. The alignment follows the Jukes-Cantor nucleotide 
subsititution model \cite{cantor1969mammalian}, 
with per-nucleotide rate of mutation, $r$.

The Jukes-Cantor
nucleotide substitution model is regarded as an oversimplification of 
nature \cite{unknown}. In this in silico experiment, however, we simulate first
a DNA alignment from a phylogeny using nucleotide substitution model $M_N$, 
then infer a phylogeny from that DNA alignment, assuming nucleotide substitution model $M_N$.
Too reduce unnecessary complexity, $M_N$ is chosen to be the simplest possible.

The combination of $l_a$ and $r$ is tuned to result in informative alignments. 
Within a maximally informative alignment, each species has its own unique DNA sequence.
Additionally, the Jukes-Cantor distance \cite{unknown} between any 
two sequences must be calculable, which is when the proportion of matching nucleotides is below 75\%.
The tuning was done using an $l_a$ of $10^3$ nucleotides, 
as this appears to be the current median of the sequenced genome lengths, as currently published.

%%%%%%%%%%%%%%%%%%%%%%%%%%%%%%%%%%%%%%%%%%%%%%%%%%%%%%%%%%%%%%%%%%%%%%%%%%%%%%%%
\subsection{Simulation of posteriors}
%%%%%%%%%%%%%%%%%%%%%%%%%%%%%%%%%%%%%%%%%%%%%%%%%%%%%%%%%%%%%%%%%%%%%%%%%%%%%%%%

From each DNA alignment, $n_b$ posteriors of phylogenies and parameter combinations 
is constructed, using BEAST2, a Bayesian phylogenetic tool \cite{bouckaert2014beast}.
The nucleotide substitution model is set to 
be Jukes-Cantor (one nucleotide substition rate category, no invariant nucleotides) 
and a strict molecular clock, which matches the procedure the DNA alignments are generated, 
as described in section \ref{subsec:simulating_alignments}).
The tree prior is set to the (non-protracted) birth-death model, 
with an uniform prior for birth-rate of range $[0, 10^5]$ and 
a uniform death-per-birth rate prior of range $[0, 1]$. 
The MCMC creation parameters were set to have $n_s$ states, sampled at an interval of $i_s$ states.

An MCMC must have an effective sample size (ESS) of at least 200 to result in a 
posterior that is 'adequate' \cite{drummond2015bayesian}. Due to the multiple 
sources of stochasticity in the preceding steps, this ESS may not always be achieved.

%%%%%%%%%%%%%%%%%%%%%%%%%%%%%%%%%%%%%%%%%%%%%%%%%%%%%%%%%%%%%%%%%%%%%%%%%%%%%%%%
\subsection{Analysis}
%%%%%%%%%%%%%%%%%%%%%%%%%%%%%%%%%%%%%%%%%%%%%%%%%%%%%%%%%%%%%%%%%%%%%%%%%%%%%%%%

To quantify the error, each species tree is compared with the posterior derived
from its $n_a$ alignments (for example, B1 and D1 in figure \ref{fig:experiment}).
This is by calculating the nLTT statistic \cite{janzen2015approximate}
between the species tree and each of its posterior's trees (using 
the nLTT package \cite{nLTT}), resulting in an nLTT statistic distribution. 

% BD trees

The nLTT statistic distribution for all simulations with a high speciation 
completion rate is visualized for the different parameter values. This allows
to see the baseline error. Instead of using a complete
nLTT statistic distribution, also the mean value of each posterior is 
sometimes used to reduce variance in this.

The effect of speciation completion time $\lambda$ is visualized, by showing
nLTT statistic distributions for the different parameter values, for all values
of $\lambda$. 

Instead of using the parameter value of $\lambda$ as a proxy for protractedness,
the speciation completion time, $t_{sc}$, and average duration of speciation, $t_{\mean{ds}}$, 
are used. Speciation completion time, $t_{sc}$, is simply the inverse of the
speciation completion rate $\lambda$:

\begin{equation}
  t_{sc} = \frac{1}{\lambda}
  \label{eq:speciation_completion_time}
\end{equation}

The speciation completion time differs from the observed average speciation time, 
as incipient species may go extinct and may produce other incipient species 
that complete speciation before the original incipient one. The average duration 
of speciation, $t_{\mean{ds}}$, takes this into 
account \todo{check formula ?19 of?20 in Etienne et al., 2012, then cite}:

\begin{equation}
    t_{\mean{ds}} = \mean{\rho} = \frac{2}{ (D - \lambda + b_i - \mu_i) \cdot \ln{ \frac{2}{(1 + (\lambda - b_i + \mu_i)/D)}  } }
\end{equation}

where $D$ is defined as:

\begin{equation}
    D = \sqrt{ (\lambda + b_i)^2 + 2 \cdot (\lambda - b_i) \cdot \mu_i + \mu_i^2 }
\end{equation}

For each simulation, the expected $t_{\mean{ds}}$ is calculated from its parameter 
values (using the PBD package \cite{PBD}), after which the nLTT statistic 
distributed is plotted, to see the effect of protractedness on the nLTT statistic
distribution.

As not all simulations result in posteriors with an adequate ESS, the
effect that a below-median or above-median ESS has on the nLTT statistic
distribution is displayed. 

To view the effect of alignment length, $l_a$, the nLTT statistic
distributions are plotted side-by-side for all other parameter values. 

%%%%%%%%%%%%%%%%%%%%%%%%%%%%%%%%%%%%%%%%%%%%%%%%%%%%%%%%%%%%%%%%%%%%%%%%%%%%%%%%
\subsection{Peripherals}
%%%%%%%%%%%%%%%%%%%%%%%%%%%%%%%%%%%%%%%%%%%%%%%%%%%%%%%%%%%%%%%%%%%%%%%%%%%%%%%%

Simulating incipient species trees, sampling species trees, simulating alignments
and creating posteriors, are done with the wiritttes package \cite{wiritttes}. The
analysis is done with the wiritttea package \cite{wiritttea}. Those packages rely on 
functionality in the APE \cite{APE}, geiger \cite{GEIGER}, ggplot2 \cite{ggplot2}, 
RBeast \cite{RBeast}. Quality of these packages is assured by 
the covr \cite{covr}, devtools \cite{devtools}, 
goodpractice \cite{goodpractice}, lintr \cite{lintr}, testit \cite{testit} 
and testthat \cite{testthat} packages, using the Travis CI \cite{travis} continuous integration service. 
Both packages are hosted at GitHub \cite{github}, a good practice for computational
scientists \cite{perez2016ten} and helps improve transparency \cite{gorgolewski2016practical}.
Code coverage, which correlates with code quality \cite{del1995correlation}, 
is above 90\% for both packages.

The simulated dataset can be downloaded from DataDyrad [but not yet]. The
summarized dataset can be downloaded at \cite{wirittte_data}. The
pre-publication of this article can be downloaded at \cite{wirittte_article}. 

Simulations and analyis was mostly performed on the Peregrine 
high performance computing cluster of the University of Groningen.
